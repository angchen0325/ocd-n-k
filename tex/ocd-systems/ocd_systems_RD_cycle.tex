% \begin{abstract}
% 开发DBO量测设备
% \end{abstract}

\newpage

\section{光学量测机台的开发流程~\label{光学量测机台的开发流程}}
光学量测机台的研发周期通常需要$2.5\sim4$年,具体取决于产品的复杂性和技术要求。
一款光学量测设备,特别是用于纳米结构的特征尺寸(Critical Dimension, CD)、厚度、折射率等精密参数测量的系统,
其完整研发周期如下:
\begin{enumerate}
\setlength{\itemsep}{2pt}
\setlength{\parsep}{0pt}
\setlength{\parskip}{0pt}
\item 原理验证机:Proof-of-Concept, POC 
\item 光学验证样机:Optical Feasibility Bench
\item Alpha样机:功能集成型
\item Beta样机:可重复可生产化雏形
\item 预量产样机:Pilot机
\item 量产机台:Mass Production, MP
\end{enumerate}
上述六个阶段又可以划分成两个大阶段:
\begin{enumerate}
\setlength{\itemsep}{2pt}
\setlength{\parsep}{0pt}
\setlength{\parskip}{0pt}
\item 第一大阶段:从\texttt{0}到\texttt{1} (原理验证机$\rightarrow$光学验证样机$\rightarrow$ Alpha样机)
\item 第二大阶段:从\texttt{1}到\texttt{100} (Beta样机$\rightarrow$预量产样机$\rightarrow$量产机台)
\end{enumerate}


\subsection{各阶段机台的功能和特点~\label{各阶段机台的功能和特点}}
原理验证机的功能和特点见表~\ref{table:POC}:
\begin{table}[h!]
\centering
\caption{\textbf{阶段1-原理验证机} (约3$\sim$6个月)}
\label{table:POC}
\begin{tabular}{l|l}
\hline\hline
\textbf{阶段名称} & \textbf{含义} \\ \hline
目标 & 验证关键物理原理,明确目标参数,竞争对手分析 \\ \hline
输出 & 基础散射谱数据$+$初步仿真$+$手动装配测试装置(非封装) \\ \hline
特点 & 非自动化、不稳定,仅验证光学或算法是否“可测” \\ \hline
参与人员 & 光学工程师、算法工程师、1名技术负责人 (3人) \\ \hline
典型设备形态 & 面包板结构$+$三轴架$+$旋钮$+$光谱仪$+$手动样品台 \\
\hline\hline
\end{tabular}
\end{table}

光学验证样机的功能和特点见表~\ref{table:OFB}:
\begin{table}[h!]
\centering
\caption{\textbf{阶段2-光学验证样机} (约4$\sim$8个月)}
\label{table:OFB}
\begin{tabular}{l|l}
\hline\hline
\textbf{阶段名称} & \textbf{含义} \\ \hline
目标 & 优化光源、探测路径与光机布局,结合仿真提升谱图质量与拟合能力) \\ \hline
输出 & 准稳定光学架构、测量精度验证报告、与SEM/AFM对比误差 \\ \hline
新增特点 & 模拟数据库构建、反演算法初步建立,开始验证不同结构可区分性 \\ \hline
参与人员 & 光学+算法+机械+软件工程师(4$\sim$5人) \\ \hline
风险点 & 样品漂移、入射角控制精度、建模误差放大等 \\
\hline\hline
\end{tabular}
\end{table}

Alpha样机的功能和特点见表~\ref{table:alpha}:
\begin{table}[h!]
\centering
\caption{\textbf{阶段3-Alpha样机} (约8$\sim$12个月)}
\label{table:alpha}
\begin{tabular}{l|l}
\hline\hline
\textbf{阶段名称} & \textbf{含义} \\ \hline
目标 & 实现功能完整系统,具备基本自动化、GUI自动扫描、离线反演等能力 \\ \hline
输出 & Alpha原型机 $+$ TRR报告 $+$ 全系统图 $+$ 结构重建与偏差分析 \\ \hline
特点 & 自动偏振调制、角度扫描、反演建模、基本算法$+$机械$+$软件融合 \\ \hline
参与人员 & 可在上阶段基础上增加1$\sim$2人(6$\sim$7人)  \\ \hline
可测参数 & CD,厚度,n/k  \\ \hline
风险点 & 机械重复性、软件控制、系统热稳定性等 \\
\hline\hline
\end{tabular}
\end{table}




% 硅(Si)、碳化硅(SiC)和氮化镓(GaN)功率芯片调节各种消费品、工业设备和最新电动汽车(Electric Vehicle, EV)的电压和电流。
% 必须监控制造过程,以确保满足功率设备类型的最高质量和可靠性标准,包括MOSFET、Power MOS、IGBT (Insulated Gate Bipolar Transistor)、BIPOLAR-CMOS-DMOS(BCD)、超级结、二极管和整流器。

% \texttt{n\&k}公司~\cite{nktechnology}的Olympian和OptiPrime(X)系列光学量测机台可以用来监测功率器的制造流程,比较典型的应用包括:
% \begin{itemize}[leftmargin=*] 
% \setlength{\itemsep}{2pt}
% \setlength{\parsep}{0pt}
% \setlength{\parskip}{0pt}
% \item 原子级(埃级)精度控制的沉积、扩散、外延和光刻工艺。
% \item 聚氧化物凹槽、触点、沟渠隔离(STI和DTI)的蚀刻结构。
% \item 高长宽比($>20:1$)沟槽和通孔,以及侧壁角度监测。
% \item 用于晶圆的SOI (Silicon-On-Insulator)和GOS (GaN-On-Silicon)的工程基质层。
% \end{itemize}




