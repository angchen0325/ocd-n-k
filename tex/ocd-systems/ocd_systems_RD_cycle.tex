% \begin{abstract}
% 开发DBO量测设备
% \end{abstract}

\newpage

\section{光学量测机台的开发流程~\label{光学量测机台的开发流程}}
光学量测机台的研发周期通常需要$2.5\sim4$年,具体取决于产品的复杂性和技术要求。
一款光学量测设备,特别是用于纳米结构的特征尺寸(Critical Dimension, CD)、厚度、折射率等精密参数测量的系统,
其完整研发周期如下:
\begin{enumerate}
\setlength{\itemsep}{2pt}
\setlength{\parsep}{0pt}
\setlength{\parskip}{0pt}
\item 原理验证机:Proof-of-Concept, POC;关键物理机制验证 
\item 光学验证样机:Optical Feasibility Bench;构建可重复性光路与仿真验证平台
\item Alpha样机:功能集成型;全系统功能集成原型
\item Beta样机:可重复可生产化雏形;结构/软件稳定性与客户可用性
\item 预量产样机:Pilot机;3$\sim$10台系统构建与交付验证
\item 量产机台:Mass Production, MP;标准化量产与售后服务体系建立
\end{enumerate}
上述六个阶段又可以划分成两个大阶段:
\begin{enumerate}
\setlength{\itemsep}{2pt}
\setlength{\parsep}{0pt}
\setlength{\parskip}{0pt}
\item 第一大阶段:从\texttt{0}到\texttt{1} (原理验证机$\rightarrow$光学验证样机$\rightarrow$ Alpha样机)
\item 第二大阶段:从\texttt{1}到\texttt{100} (Beta样机$\rightarrow$预量产样机$\rightarrow$量产机台)
\end{enumerate}


\subsection{各阶段机台的功能和特点~\label{各阶段机台的功能和特点}}
原理验证机的功能和特点见表~\ref{table:POC}:
\begin{table}[h!]
\centering
\caption{\textbf{阶段1-原理验证机} (约3$\sim$6个月)}
\label{table:POC}
\begin{tabular}{l|l}
\hline\hline
\textbf{阶段名称} & \textbf{含义} \\ \hline
目标 & 验证关键物理原理,明确目标参数,竞争对手分析 \\ \hline
输出 & 基础建模能力,手动装配测试装置(非封装),竞对产品报告\\ \hline
特点 & 非自动化、不稳定,仅验证光学或算法是否“可测” \\ \hline
参与人员 & 光学工程师、算法工程师、1名技术负责人 (3人) \\ \hline
典型设备形态 & 面包板结构$+$三轴架$+$旋钮$+$光谱仪$+$手动样品台 \\
\hline\hline
\end{tabular}
\end{table}

光学验证样机的功能和特点见表~\ref{table:OFB}:
\begin{table}[h!]
\centering
\caption{\textbf{阶段2-光学验证样机} (约4$\sim$8个月)}
\label{table:OFB}
\begin{tabular}{l|l}
\hline\hline
\textbf{阶段名称} & \textbf{含义} \\ \hline
目标 & 优化光源、探测路径与光机布局,结合仿真提升谱图质量与拟合能力) \\ \hline
输出 & 准稳定光学架构、测量精度验证报告、与SEM/AFM对比误差 \\ \hline
新增特点 & 模拟数据库构建、反演算法初步建立,开始验证不同结构可区分性 \\ \hline
参与人员 & 光学+算法+机械+软件工程师(4$\sim$5人) \\ \hline
风险点 & 样品漂移、入射角控制精度、建模误差放大等 \\
\hline\hline
\end{tabular}
\end{table}

Alpha样机的功能和特点见表~\ref{table:alpha}:
\begin{table}[h!]
\centering
\caption{\textbf{阶段3-Alpha样机} (约8$\sim$12个月)}
\label{table:alpha}
\begin{tabular}{l|l}
\hline\hline
\textbf{阶段名称} & \textbf{含义} \\ \hline
目标 & 实现功能完整系统,具备基本自动化、GUI自动扫描、离线反演等能力 \\ \hline
输出 & Alpha原型机 $+$ TRR报告 $+$ 全系统图 $+$ 结构重建与偏差分析 \\ \hline
特点 & 自动偏振调制、角度扫描、反演建模、基本算法$+$机械$+$软件融合 \\ \hline
参与人员 & 可在上阶段基础上增加1$\sim$2人(6$\sim$7人)  \\ \hline
可测参数 & CD,厚度,n/k  \\ \hline
风险点 & 机械重复性、软件控制、系统热稳定性等 \\
\hline\hline
\end{tabular}
\end{table}

Beta样机的功能和特点见表~\ref{table:beta}:
\begin{table}[h!]
\centering
\caption{\textbf{阶段4-Beta样机} (约8$\sim$10个月)}
\label{table:beta}
\begin{tabular}{l|l}
\hline\hline
\textbf{阶段名称} & \textbf{含义} \\ \hline
目标 & 机械结构稳定、软件稳定、可靠性初步验证,可在研发环境内24/7运行 \\ \hline
输出 & Beta样机 + 对比量测报告 + 可靠性测试报告(温飘、连续测量误差) \\ \hline
新增特性 & 支持校准流程(SOP)、结构更紧凑、开发对外接口(SECS/GEM预留) \\ \hline
参与人员 & 引入FAE工程师,测试工程师 (8$\sim$10人)  \\ \hline
客户参与 & 邀请早期客户评测(early access),获取工艺线反馈  \\ \hline
风险点 & 软件bug、量测不稳定、校准流程不健全等 \\
\hline\hline
\end{tabular}
\end{table}

预量产样机的功能和特点见表~\ref{table:pilot}:
\begin{table}[h!]
\centering
\caption{\textbf{阶段5-预量产样机} (约6$\sim$8个月)}
\label{table:pilot}
\begin{tabular}{l|l}
\hline\hline
\textbf{阶段名称} & \textbf{含义} \\ \hline
目标 & 批量制造流程验证、产测SOP验证、首批客户现场安装测试 \\ \hline
输出 & 小批量(3$\sim$10台)生产样机,包含完整产品文档包 \\ \hline
新增特性 & 建立工程变更流程(ECR/ECO)、建立技术支持体系(TSE) \\ \hline
参与人员 & 引入工程支持、品控、文档 (10$+$人)  \\ \hline
典型测试 & 50$+$小时连续稳定测量、$\pm$0.5 nm重复性、$\pm$1.5 nm准确度 \\ \hline
风险点 & 工程变更未闭环、供应链问题、产线反馈滞后 \\
\hline\hline
\end{tabular}
\end{table}

量产机的功能和特点见表~\ref{table:mp}:
\begin{table}[h!]
\centering
\caption{\textbf{阶段5-量产机} (约6个月)}
\label{table:mp}
\begin{tabular}{l|l}
\hline\hline
\textbf{阶段名称} & \textbf{含义} \\ \hline
目标 & 满足出厂测试、客户出货、可服务、可升级等要求 \\ \hline
性能目标 & 符合客户spec,误差$<\pm1$ nm,良率$>98$\%,$\text{MTBF}>3000$小时 \\
\hline\hline
\end{tabular}
\end{table}


\subsection{第一大阶段的WBS~\label{第一大阶段的WBS}}
我们需要确定主要的工作分解结构(Work Breakdown Structure, WBS):包括各个阶段的子任务和输出文档/交付物。

原理验证机的WBS见表~\ref{table:POC_WBS}:
\begin{table}[h!]
\centering
\caption{\textbf{原理验证机的WBS}}
\label{table:POC_WBS}
\begin{tabular}{c|l|l}
\hline\hline
\textbf{任务编号} & \textbf{子任务} & \textbf{输出文档/交付物}  \\ \hline
1.1 & 目标结构选择与分析 & 技术选型报告 \\ \hline
1.2 & 光学建模与仿真(RCWA) & 初版模拟库(.npz/.mat格式) \\ \hline
1.3 & 实验装置搭建(非集成) & 手动平台,激光器,探测器 \\ \hline
1.4 & 初步反演算法编写 & Python/Jupyter分析代码 \\ \hline
1.5 & 可测性验证与评估 & 原理验证报告(是否可量测)\\
\hline\hline
\end{tabular}
\end{table}

光学验证样机的WBS见表~\ref{table:OFB_WBS}:
\begin{table}[h!]
\centering
\caption{\textbf{光学验证样机的WBS}}
\label{table:OFB_WBS}
\begin{tabular}{c|l|l}
\hline\hline
\textbf{任务编号} & \textbf{子任务} & \textbf{输出文档/交付物}  \\ \hline
2.1 & 光源/光路系统搭建 & Zemax/FRED光路图、波前仿真报告 \\ \hline
2.2 & 样品测试实验& 实测谱图(光谱$\leftrightarrow$结构) \\ \hline
2.3 & 多角度扫描 + 偏振控制实验 & 散射角度映射图 \\ \hline
2.4 & 模拟谱与实测谱对比分析 & 曲线拟合报告、$R^2$评价指标 \\ \hline
2.5 & 下一阶段设计草案 & Alpha架构建议图、风险点列表 \\
\hline\hline
\end{tabular}
\end{table}

Alpha样机的WBS见表~\ref{table:alpha_WBS}:
\begin{table}[h!]
\centering
\caption{\textbf{Alpha样机的WBS}}
\label{table:alpha_WBS}
\begin{tabular}{c|l|l}
\hline\hline
\textbf{任务编号} & \textbf{子任务} & \textbf{输出文档/交付物}  \\ \hline
3.1 & 系统功能模块设计(光+机+电) & 模块图纸、接口协议、控制框架图 \\ \hline
3.2 & 软件平台开发(GUI + 数据分析) & 界面原型图、操作流程文档 \\ \hline
3.3 & 样品量测与误差分析 & 误差统计表、重复性报告 \\ \hline
3.4 & TRR测试评估 & 技术成熟度评审表、样机照片、谱图 \\ \hline
3.5 & 迭代优化建议总结 & Bug list + Alpha改进建议 \\
\hline\hline
\end{tabular}
\end{table} 

这里,TRR (Technical Readiness Review)指的是技术成熟度评估;
在研发中,制定TRR测试计划至关重要。