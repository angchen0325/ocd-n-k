% \begin{abstract}
% 开发DBO量测设备
% \end{abstract}

\newpage

\section{简介~\label{简介}}
曝光显影后存留在光刻胶上的图形[被称为当前层(current layer)]必须与晶圆衬底上已个的图[被称为参考层(reference layer)]对准~\cite{yayi2016super},
这样才能保证器件各部分之间连接正确。
对准误差太大是导致器件短路和断路的上要原因之一,它极大地影响器件的良率。

在集成屯路制造的流程中,有专门的设备通过测量品圆上当前图形(光刻胶图形)与参考图形(衬底内图形)之间的相对位四米确定套刻的误差(overlay)。
套刻误差定量地描述了当前的图形相对于参考图形沿X和Y方向的偏差,以及这种偏并在晶圆表面的分布。
与图形线宽(CD)一样,套刻误差也是监测光刻工艺好坏的个关键指标。
理想的情况是当前层与参考层的图形正对准,即套刻误差是零。

在光刻工艺中,套刻误差是通过以下三部分的协同工作来减小的:
\begin{enumerate}[leftmargin=*] 
\setlength{\itemsep}{2pt}
\setlength{\parsep}{0pt}
\setlength{\parskip}{0pt}
\item 光刻机的\CJKtextit{对准}系统:晶圆被放置在光刻机的晶圆工件台上后,光刻机的对准传感器(alignment sensor)测量晶圆的位置,
和掩模(mask)上的图形对准,实施曝光。
\item 完成显影后,晶圆被传送到\CJKtextit{量测}设备(overlay metrology tool);在这里,光刻胶图形和参考层图形之间的套刻误差被精确测定。
\item 套刻误差数据被上传到专用软件中,软件根据事先设定的\CJKtextit{模型}做计算,算出套刻误养中可以被修正的量,
然后这些修正量被\CJKtextit{反馈}到光刻机的对准系统,对曝光位置做进一步步的修正(在对准测量的基础上)。
\end{enumerate}
另外,特别区分一下\CJKtextit{对准}和\CJKtextit{套刻误差}这两个概念:
\begin{itemize}[leftmargin=*] 
\setlength{\itemsep}{2pt}
\setlength{\parsep}{0pt}
\setlength{\parskip}{0pt}
\item \CJKtextit{对准}是指测定晶圆上参考层图形的位置并调整曝光系统,使当前曝光的图形与晶圆上的图形精确重叠的过程;
对准操作是由光刻机中的对准系统来完成的。
\item \CJKtextit{套刻误差}是衡量对准好坏的参数,它直接定量描述当前层与考层之间的位置偏差;套刻误差由专用量测设备测量得到。
\end{itemize}

目前业界使用的套刻误差量测有两种方式:
如果是通过在光学显微镜下对比图形位置的偏差来实现的,这种方法基于图像识别技术,被称为IBO~(image based overlay);
如果是通过光学衍射的原理来测量实现的,则被称为DBO~(diffraction based overlay)。

这里我们主要研发基于DBO的量测设备~\cite{adel2008diffraction},同时也会结合IBO设备中的原理和技术(详情见~\ref{成像套刻量测}~节)。

\subsection{DBO量测的重要性~\label{DBO量测的重要性}}
随着半导体工艺技术不断向先进节点发展,套刻测量在精密制造中的作用变得愈加重要。
在先进节点工艺中,DBO 技术展现了诸多独特优势:
\begin{itemize}[leftmargin=*] 
\setlength{\itemsep}{2pt}
\setlength{\parsep}{0pt}
\setlength{\parskip}{0pt}
\item 首先,与电子显微镜技术不同,DBO 是一种非破坏性测量方法,其对晶圆表面或光刻
胶等敏感材料无任何物理损伤,使其更适合用于在线测量和高精度过程控制。
\item 其次,DBO 技术具有极高的测量速率,使其能够实现密集的采样覆盖。这种高密度测
量不仅能够深入到复杂电路图形的内部,还能够更加准确地反映电路实际的几何和套刻特性,为精准的工艺优化提供可靠数据支持。
\end{itemize}

因此,随着节点尺寸的进一步缩小,DBO 技术凭借其非破坏性、高速和高精度的测量特性,成为先进节点工艺中不可或缺的关键计量手段。
这一技术的发展和应用无疑是半导体制造行业发展的必然趋势。

在先进节点工艺中,由于层次复杂、衬底材料多样、前序制造工艺差异显著,以及光刻胶和照明条件的多变性,传统的研发方式面临巨大挑战。
如果采用“设计–流片验证–优化设计–再次流片验证”的循环模式,不仅成本高昂,还会导致研发周期过长,难以满足先进节点对效率和经济性的严格要求。
此外,直接依赖最终版光罩的流片实验来收集测量数据并优化工艺的方式,已经无法适应先进节点工艺对快速研发和成本控制的需求。
因此,\CJKtextit{借助计算型光刻技术,通过高精度的物理模型模拟特定制造条件下的测量结果},成为取代部分流片实验的有效方法。
此种方法不仅大幅降低了研发中的物质成本,还显著缩短了研发周期。目前,业界领军企业如台积电和三星等,已广泛采用D4C (Design for Control) 方法,成功开发了新一代套刻误差测量系统,为先进制造工艺的推进提供了强有力的技术支持。

在先进节点工艺的关键层次上,开发符合先进集成电路制造工艺要求的高性能测量系统是至关重要的。
该系统需在保持高测量灵敏度的同时,进一步提升测量精度并降低测量误差。
通过将高精度的测量结果反馈至曝光控制系统,可显著增强套刻精度的准确性和稳定性,从而有效提升工艺良率,为先进节点的高质量制造提供可靠保障。

\subsection{影响套刻误差量测精度的因素~\label{影响套刻误差量测精度的因素}}

套刻误差的测量精确性受到工艺流程中前续制造工艺的影响,主要的影响因素有四个:
\begin{enumerate}[leftmargin=*] 
\setlength{\itemsep}{2pt}
\setlength{\parsep}{0pt}
\setlength{\parskip}{0pt}
\item 化学机械抛光(Chemical Mechanical Polishing, CMP)
\item 刻蚀(Etching)
\item 薄膜沉积工艺(Thin Film Deposition)
\item 光刻工艺(Photolithography)
\end{enumerate}